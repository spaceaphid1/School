\PassOptionsToPackage{unicode=true}{hyperref} % options for packages loaded elsewhere
\PassOptionsToPackage{hyphens}{url}
%
\documentclass[]{article}
\usepackage{lmodern}
\usepackage{amssymb,amsmath}
\usepackage{ifxetex,ifluatex}
\usepackage{fixltx2e} % provides \textsubscript
\ifnum 0\ifxetex 1\fi\ifluatex 1\fi=0 % if pdftex
  \usepackage[T1]{fontenc}
  \usepackage[utf8]{inputenc}
  \usepackage{textcomp} % provides euro and other symbols
\else % if luatex or xelatex
  \usepackage{unicode-math}
  \defaultfontfeatures{Ligatures=TeX,Scale=MatchLowercase}
\fi
% use upquote if available, for straight quotes in verbatim environments
\IfFileExists{upquote.sty}{\usepackage{upquote}}{}
% use microtype if available
\IfFileExists{microtype.sty}{%
\usepackage[]{microtype}
\UseMicrotypeSet[protrusion]{basicmath} % disable protrusion for tt fonts
}{}
\IfFileExists{parskip.sty}{%
\usepackage{parskip}
}{% else
\setlength{\parindent}{0pt}
\setlength{\parskip}{6pt plus 2pt minus 1pt}
}
\usepackage{hyperref}
\hypersetup{
            pdftitle={Lab 1: Introduction to R},
            pdfauthor={Jackson Anderson},
            pdfborder={0 0 0},
            breaklinks=true}
\urlstyle{same}  % don't use monospace font for urls
\usepackage[margin=1in]{geometry}
\usepackage{color}
\usepackage{fancyvrb}
\newcommand{\VerbBar}{|}
\newcommand{\VERB}{\Verb[commandchars=\\\{\}]}
\DefineVerbatimEnvironment{Highlighting}{Verbatim}{commandchars=\\\{\}}
% Add ',fontsize=\small' for more characters per line
\usepackage{framed}
\definecolor{shadecolor}{RGB}{248,248,248}
\newenvironment{Shaded}{\begin{snugshade}}{\end{snugshade}}
\newcommand{\AlertTok}[1]{\textcolor[rgb]{0.94,0.16,0.16}{#1}}
\newcommand{\AnnotationTok}[1]{\textcolor[rgb]{0.56,0.35,0.01}{\textbf{\textit{#1}}}}
\newcommand{\AttributeTok}[1]{\textcolor[rgb]{0.77,0.63,0.00}{#1}}
\newcommand{\BaseNTok}[1]{\textcolor[rgb]{0.00,0.00,0.81}{#1}}
\newcommand{\BuiltInTok}[1]{#1}
\newcommand{\CharTok}[1]{\textcolor[rgb]{0.31,0.60,0.02}{#1}}
\newcommand{\CommentTok}[1]{\textcolor[rgb]{0.56,0.35,0.01}{\textit{#1}}}
\newcommand{\CommentVarTok}[1]{\textcolor[rgb]{0.56,0.35,0.01}{\textbf{\textit{#1}}}}
\newcommand{\ConstantTok}[1]{\textcolor[rgb]{0.00,0.00,0.00}{#1}}
\newcommand{\ControlFlowTok}[1]{\textcolor[rgb]{0.13,0.29,0.53}{\textbf{#1}}}
\newcommand{\DataTypeTok}[1]{\textcolor[rgb]{0.13,0.29,0.53}{#1}}
\newcommand{\DecValTok}[1]{\textcolor[rgb]{0.00,0.00,0.81}{#1}}
\newcommand{\DocumentationTok}[1]{\textcolor[rgb]{0.56,0.35,0.01}{\textbf{\textit{#1}}}}
\newcommand{\ErrorTok}[1]{\textcolor[rgb]{0.64,0.00,0.00}{\textbf{#1}}}
\newcommand{\ExtensionTok}[1]{#1}
\newcommand{\FloatTok}[1]{\textcolor[rgb]{0.00,0.00,0.81}{#1}}
\newcommand{\FunctionTok}[1]{\textcolor[rgb]{0.00,0.00,0.00}{#1}}
\newcommand{\ImportTok}[1]{#1}
\newcommand{\InformationTok}[1]{\textcolor[rgb]{0.56,0.35,0.01}{\textbf{\textit{#1}}}}
\newcommand{\KeywordTok}[1]{\textcolor[rgb]{0.13,0.29,0.53}{\textbf{#1}}}
\newcommand{\NormalTok}[1]{#1}
\newcommand{\OperatorTok}[1]{\textcolor[rgb]{0.81,0.36,0.00}{\textbf{#1}}}
\newcommand{\OtherTok}[1]{\textcolor[rgb]{0.56,0.35,0.01}{#1}}
\newcommand{\PreprocessorTok}[1]{\textcolor[rgb]{0.56,0.35,0.01}{\textit{#1}}}
\newcommand{\RegionMarkerTok}[1]{#1}
\newcommand{\SpecialCharTok}[1]{\textcolor[rgb]{0.00,0.00,0.00}{#1}}
\newcommand{\SpecialStringTok}[1]{\textcolor[rgb]{0.31,0.60,0.02}{#1}}
\newcommand{\StringTok}[1]{\textcolor[rgb]{0.31,0.60,0.02}{#1}}
\newcommand{\VariableTok}[1]{\textcolor[rgb]{0.00,0.00,0.00}{#1}}
\newcommand{\VerbatimStringTok}[1]{\textcolor[rgb]{0.31,0.60,0.02}{#1}}
\newcommand{\WarningTok}[1]{\textcolor[rgb]{0.56,0.35,0.01}{\textbf{\textit{#1}}}}
\usepackage{graphicx,grffile}
\makeatletter
\def\maxwidth{\ifdim\Gin@nat@width>\linewidth\linewidth\else\Gin@nat@width\fi}
\def\maxheight{\ifdim\Gin@nat@height>\textheight\textheight\else\Gin@nat@height\fi}
\makeatother
% Scale images if necessary, so that they will not overflow the page
% margins by default, and it is still possible to overwrite the defaults
% using explicit options in \includegraphics[width, height, ...]{}
\setkeys{Gin}{width=\maxwidth,height=\maxheight,keepaspectratio}
\setlength{\emergencystretch}{3em}  % prevent overfull lines
\providecommand{\tightlist}{%
  \setlength{\itemsep}{0pt}\setlength{\parskip}{0pt}}
\setcounter{secnumdepth}{0}
% Redefines (sub)paragraphs to behave more like sections
\ifx\paragraph\undefined\else
\let\oldparagraph\paragraph
\renewcommand{\paragraph}[1]{\oldparagraph{#1}\mbox{}}
\fi
\ifx\subparagraph\undefined\else
\let\oldsubparagraph\subparagraph
\renewcommand{\subparagraph}[1]{\oldsubparagraph{#1}\mbox{}}
\fi

% set default figure placement to htbp
\makeatletter
\def\fps@figure{htbp}
\makeatother

\usepackage{etoolbox}
\makeatletter
\providecommand{\subtitle}[1]{% add subtitle to \maketitle
  \apptocmd{\@title}{\par {\large #1 \par}}{}{}
}
\makeatother

\title{Lab 1: Introduction to R}
\providecommand{\subtitle}[1]{}
\subtitle{Introduction to R, Summary Statistics of Data}
\author{Jackson Anderson}
\date{August 27, 2020}

\begin{document}
\maketitle

\begin{Shaded}
\begin{Highlighting}[]
\CommentTok{# loading packages #}
\KeywordTok{library}\NormalTok{(tidyverse)}
\end{Highlighting}
\end{Shaded}

\begin{verbatim}
## -- Attaching packages ----------------------------------------------------------------------------------- tidyverse 1.3.0 --
\end{verbatim}

\begin{verbatim}
## v ggplot2 3.2.1     v purrr   0.3.3
## v tibble  2.1.3     v dplyr   0.8.4
## v tidyr   1.0.2     v stringr 1.4.0
## v readr   1.3.1     v forcats 0.4.0
\end{verbatim}

\begin{verbatim}
## -- Conflicts -------------------------------------------------------------------------------------- tidyverse_conflicts() --
## x dplyr::filter() masks stats::filter()
## x dplyr::lag()    masks stats::lag()
\end{verbatim}

\begin{Shaded}
\begin{Highlighting}[]
\KeywordTok{library}\NormalTok{(knitr)}
\KeywordTok{library}\NormalTok{(tinytex)}
\end{Highlighting}
\end{Shaded}

\hypertarget{general-information}{%
\section{General information}\label{general-information}}

This lab is due Wednesday, September 2nd by 11:59 pm and is worth 5
points. You must upload your .rmd file and your knitted PDF to the
assignment folder on Canvas.

You are welcome and encouraged to talk with classmates and ask for help.
However, each student should turn in their own lab assignment and all
answers, including all code, needs to be solely your own.

\hypertarget{how-to-use-the-pdf-and-r-markdown-document}{%
\section{How to use the pdf and R markdown
document}\label{how-to-use-the-pdf-and-r-markdown-document}}

Download the .pdf and .rmd documents for each week's lab and keep in the
same folder, along with any data. Then, double click to open the .Rmd
document in R studio. Enter your name and date above where it says
``Your Name Here''. The .rmd file is intended for you to edit and input
your answers. The .pdf contains the same exact information but rendered
in an easier-to-read format. I suggest reading through the pdf document,
and then switching to .rmd when you are asked to input code or give an
answer.

\hypertarget{objective}{%
\section{Objective}\label{objective}}

The goal of this lab is to familiarize yourself with using R, learn the
basics of data manipulation in R, and to explore descriptive statistics
in R.

\hypertarget{basics-of-r}{%
\section{Basics of R}\label{basics-of-r}}

R is a versatile computer language that operates as both an excellent
statistical package and a programming language. In actuality, the lines
between these will blur for us, as we will do our own programming but
also will explore built-in statistical packages once we have the
conceptual frameworks mastered. R has become the `computer language for
ecology,' meaning that knowing R will help you reproduce and modify
others' results (as R code is commonly published online with papers) as
well as in future labs.

At a very basic level, R can operate as a calculator. When you open
RStudio, your screen will likely be split into four panels. At the
bottom is the \emph{console} and at the top, you have your R Markdown or
R Scripts open. In the console, you will see a command prompt, \(>\). Go
ahead and type in an equation, say \(2 + 2\), and hit ENTER. You see
that R prints the answer, just like using a calculator. Now, try typing
in some other basic arithmetic commands, like \(2 - 2\), 2 * 4, \(4/2\),
and 4 \^{} 2.

For reproducibility, it is best to type all non-exploratory code in an R
Script or R Markdown file. That way, you can go back and follow each
step. For labs, all final code must be written in the R Markdown
document. Otherwise, it will be impossible to know what you did. To do
so, we can type code directly into the R Markdown document. To separate
code from text, we use the following syntax: \(`\)\(`\)\(`\)\{r\} (three
backquotes, curly brace, r, curly brace) begins a part of code and
\(`\)\(`\)\(`\) denotes the end of code. When you export (or ``knit'')
the .rmd to a PDF, the code is evaluated and output is displayed.

To insert chunks, you can type the \(`\)\(`\)\(`\)\{r\} by hand, or go
up to the menu and select Code -\textgreater{} Insert Chunk.

To write comments on your code you can use the hashtag or pound symbol.
Whatever is typed after the hashtag will be ignored by R's computer.
This is helpful for describing what each line of your code does.

\begin{Shaded}
\begin{Highlighting}[]
\CommentTok{# code goes here. We'll use the 2 + 2 example}
\DecValTok{2} \OperatorTok{+}\StringTok{ }\DecValTok{2}
\end{Highlighting}
\end{Shaded}

\begin{verbatim}
## [1] 4
\end{verbatim}

Note that the answer to this math is included as output in the Rmd file
(below the chunk of R code) and in the console below.

Note also that the \(`\)\(`\)\(`\)\{r\} and \(`\)\(`\)\(`\) do not
render (print) in the pdf, but rather tell R that this is a code
section. The pound sign is used in code to denote comments. This helps
you (or others) follow your logic. When you enter code into an R
Markdown document or R Script, to evaluate the code, you can place your
cursor in the line of code you want to evaluate and then press
Ctrl+Enter (Windows) or Command+Return (Mac). You do not need to
``knit'' your R markdown into PDF each time you evaluate code.

Practice writing some simple arithmetic equations in the above code
chunk and evaluating them by running the code.

\hypertarget{using-the-assignment-operator}{%
\section{Using the assignment
operator}\label{using-the-assignment-operator}}

Instead of just typing in an equation, say \(4 * 8\), you will often
want to save the output value to use later on. You do so using the
\emph{assignment operator}, \textless{}-. For example,

\begin{Shaded}
\begin{Highlighting}[]
\NormalTok{x <-}\StringTok{ }\DecValTok{4} \OperatorTok{*}\StringTok{ }\DecValTok{8}
\end{Highlighting}
\end{Shaded}

The solution to the equation \(4*8\) is saved as a variable called x.
You could substitute ``x'' for any other letter or variable name you
like. When you assign variables using \textless{}-, the answer is not
automatically printed, but is rather stored as an \emph{object}. In the
top right window in RStudio there is a tab called `Environment'. If you
click on this tab you should be able to see your new object called ``x''
as well as it's value. This `Environment' is called your workspace. You
can easily see the answer by just typing in \(x\) to the console or on
your R Markdown sheet.

Sometimes you will see the \(=\) sign used instead of \textless{}-. They
are equivalent in most cases, though most R users agree that
\textless{}- should be used for assigning variables.

If you want to see the value of a variable you just assigned, just type
the name of the variable and hit enter. This is also called ``printing''
the variable. You will often be asked to print/display certain values in
your assignments, meaning you should put the name of that value in your
code chunk.

For instance, a question might ask you to multiply 8 and 4 and save that
value as the variable ``x''. Then, I might ask you to print the value of
x (i.e., the answer).

\begin{Shaded}
\begin{Highlighting}[]
\NormalTok{x <-}\StringTok{ }\DecValTok{4} \OperatorTok{*}\StringTok{ }\DecValTok{8}
\NormalTok{x }\CommentTok{# prints the value now saved as x }
\end{Highlighting}
\end{Shaded}

\begin{verbatim}
## [1] 32
\end{verbatim}

\begin{Shaded}
\begin{Highlighting}[]
\NormalTok{a =}\StringTok{ }\DecValTok{4} \OperatorTok{*}\StringTok{ }\DecValTok{8}
\NormalTok{a }\CommentTok{# prints the value now saved as a}
\end{Highlighting}
\end{Shaded}

\begin{verbatim}
## [1] 32
\end{verbatim}

\textcolor{red}{Q1: You can  use previously assigned variables in future operations. For example try multiplying $x$ by 5 and then adding $12$ to this value. Save this answer as a new variable (call it something new!). Now take this new variable and divide it by $2.2$, again saving your answer as a new variable. Print out the value of this new variable.}

\begin{Shaded}
\begin{Highlighting}[]
\NormalTok{newValue <-}\StringTok{ }\DecValTok{5}\OperatorTok{*}\NormalTok{x }\OperatorTok{+}\StringTok{ }\DecValTok{12} 
\NormalTok{newValue2 <-}\StringTok{ }\NormalTok{newValue}\OperatorTok{/}\FloatTok{2.2}
\NormalTok{newValue2}
\end{Highlighting}
\end{Shaded}

\begin{verbatim}
## [1] 78.18182
\end{verbatim}

Note that R is case-sensitive. Try typing \(X\) on the console. What
does R return?

We don't want to store all the variables we create forever. Best
practice is to export important datasets into .csv or .txt files (or
save the R script that created them) rather than rely on stored objects
in R. It is usually a good idea to start your R coding session with a
``clean slate'' by removing variables stored from previous sessions. You
can remove an object from the workspace using the \emph{rm} function.

\begin{Shaded}
\begin{Highlighting}[]
\KeywordTok{rm}\NormalTok{(x)}
\end{Highlighting}
\end{Shaded}

Note that after running this line of code the ``x'' object is now
removed from our ``environment'' or workspace. Now, try to print the
variable x you assigned above, by typing it into your console and
hitting return. What happens?

We can also remove all of the objects in our workspace by clicking on
the `broom' symbol in the Environment panel in the top right. You can
also use the following code to remove all of the objects in your
workspace:

\begin{Shaded}
\begin{Highlighting}[]
\KeywordTok{rm}\NormalTok{(}\DataTypeTok{list=}\KeywordTok{ls}\NormalTok{())}
\end{Highlighting}
\end{Shaded}

BEWARE: This code will remove all objects in you current workspace. It
is generally not a good idea to have bits of code like this in your
files because this code could accidentally remove all of the objects in
your workspace.

Instead it is preferred to use the `broom' icon in the Environment
window.

\hypertarget{getting-help-in-r}{%
\section{Getting help in R}\label{getting-help-in-r}}

R has wonderful help and example code. If you click on \emph{Help} on
the right hand side of RStudio above the lower right panel, you can see
manuals and resources both for R and RStudio.

Additionally, if you type ? followed by an R command, R will open
documentation on the command. {[}Note: ``commands'' are codes in R that
perform some type of task, such as \emph{mean()}, \emph{sum()}, or
\emph{plot()}{]}. For example:

\begin{Shaded}
\begin{Highlighting}[]
\NormalTok{?mean}
\end{Highlighting}
\end{Shaded}

This is helpful when you can't remember all of the arguments for a
command you are using (much more on this later). Additionally, typing
\emph{example(command)}, provides example code on how to use the given
function.

Since R is open-source, there are great resources and question forums
online as well. If there is a function you think R should be able to do,
you are probably right! Google is a great source for R. Also, stack
overflow has answers to thousands of questions from R users. Likely
someone has asked a similar question that is already answered on stack
overflow.

\textcolor{red}{Q2: One of the many built-in commands in R is the square root function, sqrt(). Enter code for opening the R Help documentation on the square root function.}

\begin{Shaded}
\begin{Highlighting}[]
\NormalTok{?}\KeywordTok{sqrt}\NormalTok{()}
\end{Highlighting}
\end{Shaded}

\textcolor{red}{Q3: Now, use the square root command to find the square root of 9.999, saving the answer as variable, $sqEx$. Print the answer as well.}

\begin{Shaded}
\begin{Highlighting}[]
\KeywordTok{sqrt}\NormalTok{(}\FloatTok{9.999}\NormalTok{)}
\end{Highlighting}
\end{Shaded}

\begin{verbatim}
## [1] 3.16212
\end{verbatim}

\hypertarget{vectors-and-data-frames}{%
\section{Vectors and data frames}\label{vectors-and-data-frames}}

\hypertarget{vectors}{%
\subsection{Vectors}\label{vectors}}

Vectors, matrices, and data frames are all ways to hold multiple data
elements, called \emph{elements}, of the same type. Elements can be
numbers or character strings (i.e.~words). Vectors can be created in
multiple ways. The most common are to use the \(c()\), \(seq()\), or
\(rep()\) functions.

The \(c()\) function combines multiple elements together, while
\(seq()\) automatically lists a sequence of values, allowing you to
specify the starting value, ending value, the amount you count by, or
the number of elements you want to include in a given vector. Type
\(?seq()\) for more information. Finally, \(rep()\) allows you to repeat
an element a certain amount of times. Let's look at some examples. Run
and explore the following code:

\begin{Shaded}
\begin{Highlighting}[]
\NormalTok{ex1 <-}\StringTok{ }\KeywordTok{c}\NormalTok{ (}\DecValTok{1}\NormalTok{, }\DecValTok{5}\NormalTok{, }\DecValTok{8}\NormalTok{, }\DecValTok{9}\NormalTok{)}
\NormalTok{ex2 <-}\StringTok{ }\KeywordTok{c}\NormalTok{( }\DecValTok{1}\NormalTok{, }\DecValTok{10}\OperatorTok{:}\DecValTok{15}\NormalTok{) }\CommentTok{# What does ex2 look like? How many elements?}
\CommentTok{# remember, to see the value of something you just created, you should type the name and run it to print the answer below the code chunk:}
\NormalTok{ex2}
\end{Highlighting}
\end{Shaded}

\begin{verbatim}
## [1]  1 10 11 12 13 14 15
\end{verbatim}

\begin{Shaded}
\begin{Highlighting}[]
\CommentTok{# Multiple functions can be called at the same time.}
\NormalTok{ex3 <-}\StringTok{ }\KeywordTok{c}\NormalTok{(}\DecValTok{1}\NormalTok{, }\DecValTok{5}\NormalTok{, }\KeywordTok{seq}\NormalTok{(}\DataTypeTok{from=}\DecValTok{1}\NormalTok{, }\DataTypeTok{to=}\DecValTok{10}\NormalTok{, }\DataTypeTok{by=}\FloatTok{2.2}\NormalTok{)) }\CommentTok{# What is this code doing?}
\NormalTok{ex4 <-}\StringTok{ }\KeywordTok{c}\NormalTok{(}\KeywordTok{rep}\NormalTok{(}\DecValTok{5}\NormalTok{, }\DecValTok{3}\NormalTok{), }\KeywordTok{rep}\NormalTok{(}\DecValTok{3}\NormalTok{, }\DecValTok{5}\NormalTok{), }\KeywordTok{seq}\NormalTok{(}\DataTypeTok{from=}\OperatorTok{-}\DecValTok{5}\NormalTok{, }\DataTypeTok{to=}\DecValTok{0}\NormalTok{))}
\end{Highlighting}
\end{Shaded}

Make sure you understand the difference between \(c()\), \(seq()\), or
\(rep()\)!

To extract elements from a vector, you use square brackets, \([]\). For
example, if you want the third element from \(ex1\):

\begin{Shaded}
\begin{Highlighting}[]
\NormalTok{ex1[}\DecValTok{3}\NormalTok{]}
\end{Highlighting}
\end{Shaded}

\begin{verbatim}
## [1] 8
\end{verbatim}

Another handy function for vectors is \(length()\), which gives you the
number of elements in a vector.

\begin{Shaded}
\begin{Highlighting}[]
\KeywordTok{length}\NormalTok{(ex4)}
\end{Highlighting}
\end{Shaded}

\begin{verbatim}
## [1] 14
\end{verbatim}

\begin{Shaded}
\begin{Highlighting}[]
\CommentTok{# Or, to find the value of the last element in ex4,}
\NormalTok{ex4[}\KeywordTok{length}\NormalTok{(ex4)]}
\end{Highlighting}
\end{Shaded}

\begin{verbatim}
## [1] 0
\end{verbatim}

Note that this finds the last element in this vector because
\(length(ex4)\) is equal to \(14\) and \(ex4[14]\) just return the 14th
element in the vector. All we have done is to place one command within
another. This is a helpful feature of coding that we will use a lot.

We don't have to just use numbers when creating vectors or assigning
variables in R. We can also use characters.

\begin{Shaded}
\begin{Highlighting}[]
\NormalTok{birds <-}\StringTok{ }\KeywordTok{c}\NormalTok{(}\StringTok{"Mourning Dove"}\NormalTok{, }\StringTok{"Downy Woodpecker"}\NormalTok{,}\StringTok{"American Robin"}\NormalTok{)}
\NormalTok{weights <-}\StringTok{ }\KeywordTok{c}\NormalTok{(}\DecValTok{120}\NormalTok{, }\DecValTok{28}\NormalTok{, }\DecValTok{80}\NormalTok{)}
\CommentTok{# Note how R denotes number vs. characters}
\KeywordTok{class}\NormalTok{(birds)}
\end{Highlighting}
\end{Shaded}

\begin{verbatim}
## [1] "character"
\end{verbatim}

\begin{Shaded}
\begin{Highlighting}[]
\KeywordTok{class}\NormalTok{(weights)}
\end{Highlighting}
\end{Shaded}

\begin{verbatim}
## [1] "numeric"
\end{verbatim}

\textcolor{red}{Q4: Now, add a fourth bird to our vector of birds. If you don't know, try googling first! Hint: what element number should it be? Display the vector to make sure it worked.}

\begin{Shaded}
\begin{Highlighting}[]
\NormalTok{birds[}\DecValTok{4}\NormalTok{] <-}\StringTok{ "Red-Tailed Hawk"}
\end{Highlighting}
\end{Shaded}

\hypertarget{data-frames}{%
\section{Data Frames}\label{data-frames}}

Data frames are commonly used in statistical analyses. A dataframe is
analogous to a typical Excel table that you would use to organize your
data, with a header row and each column containing a different type of
data. Different columns can be of different classes. Therefore, in a
data frame, you can mix numbers, characters, etc. across columns, which,
you can imagine, is quite useful. For example, say you have sampled
sites across an elevational gradient and recorded abundances of
different species at each site. A nice structure for your data would
then be a column of site names (characters), a column with the elevation
for each site, followed by multiple columns of abundances of different
species.

Let's go ahead and look at look at one of R's example datasets, called
CO2. You can print the whole dataframe if you type CO2 in your console
(make sure you pay attention to capitalization!).

\textcolor{red}{Q5: Describe what is included in each column in the CO2 dataframe. Note: refer to the R help '?' to do this.}

\begin{Shaded}
\begin{Highlighting}[]
\CommentTok{# The first column, Plant, contains factor data with 12 levels. The second column, Type, contains factor data with 2 levels. The third column, Treatment, is a factor type with 2 levels. The fourth column, conc, contains numerica data, as does the 5th column, uptake}
\end{Highlighting}
\end{Shaded}

\emph{The first column, Plant, contains factor data with 12 levels. The
second column, Type, contains factor data with 2 levels. The third
column, Treatment, is a factor type with 2 levels. The fourth column,
conc, contains numerica data, as does the 5th column, uptake}

To refer to specific rows or columns of a dataframe, you will use
brackets \texttt{{[}\ ,\ {]}}. The first number in a bracket is the row
number, and the second number in a bracket is the column number. Your
bracket will always contain two slots if you are referring to a
2-dimensional dataset (e.g.~a dataframe). Since vectors are one
dimensional, we use a single number to pull out specific records (as
demonstrated above).

For example, we can ask R to return the element in the 1st row and the
4th column.

\begin{Shaded}
\begin{Highlighting}[]
\NormalTok{CO2[}\DecValTok{1}\NormalTok{, }\DecValTok{4}\NormalTok{]}
\end{Highlighting}
\end{Shaded}

\begin{verbatim}
## [1] 95
\end{verbatim}

Though you can enter a row value and a column value, if you leave one of
the slots blank, it will display all rows or all columns.

For instance, we can print just the third column of CO2. Leaving the
first argument blank means that we will display all of the rows:

\begin{Shaded}
\begin{Highlighting}[]
\NormalTok{CO2[ ,}\DecValTok{3}\NormalTok{]}
\end{Highlighting}
\end{Shaded}

\begin{verbatim}
##  [1] nonchilled nonchilled nonchilled nonchilled nonchilled nonchilled
##  [7] nonchilled nonchilled nonchilled nonchilled nonchilled nonchilled
## [13] nonchilled nonchilled nonchilled nonchilled nonchilled nonchilled
## [19] nonchilled nonchilled nonchilled chilled    chilled    chilled   
## [25] chilled    chilled    chilled    chilled    chilled    chilled   
## [31] chilled    chilled    chilled    chilled    chilled    chilled   
## [37] chilled    chilled    chilled    chilled    chilled    chilled   
## [43] nonchilled nonchilled nonchilled nonchilled nonchilled nonchilled
## [49] nonchilled nonchilled nonchilled nonchilled nonchilled nonchilled
## [55] nonchilled nonchilled nonchilled nonchilled nonchilled nonchilled
## [61] nonchilled nonchilled nonchilled chilled    chilled    chilled   
## [67] chilled    chilled    chilled    chilled    chilled    chilled   
## [73] chilled    chilled    chilled    chilled    chilled    chilled   
## [79] chilled    chilled    chilled    chilled    chilled    chilled   
## Levels: nonchilled chilled
\end{verbatim}

Or, we could look at the 40th row of CO2. In this case, we have left the
second argument blank, meaning that we will display all of the columns:

\begin{Shaded}
\begin{Highlighting}[]
\NormalTok{CO2[}\DecValTok{40}\NormalTok{, ]}
\end{Highlighting}
\end{Shaded}

\begin{verbatim}
## Grouped Data: uptake ~ conc | Plant
##    Plant   Type Treatment conc uptake
## 40   Qc3 Quebec   chilled  500   38.9
\end{verbatim}

Some other functions you will often use with dataframes:

\begin{itemize}
\item
  Print the names of each column in a data frame using the function
  \texttt{names()}.
\item
  Access a single column in a dataframe using the dollar symbol to give
  the column name (instead of using the column number).
\item
  Preview the first few rows of a data frame using the function
  \texttt{head()}.
\item
  See the structure of the dataset using the \texttt{str()} command.
  This command is a useful first step to explore datasets!
\item
  Perform calculations on certain columns, such as \texttt{mean()},
  \texttt{min()}, \texttt{max()}, \texttt{sd()}. Count the number of
  records in a column using \texttt{length()}.
\end{itemize}

For example:

\begin{Shaded}
\begin{Highlighting}[]
\CommentTok{# First, let's rename CO2, since we'll want to manipulate it ourselves. }
\NormalTok{CO2ex <-}\StringTok{ }\NormalTok{CO2}
\end{Highlighting}
\end{Shaded}

\begin{Shaded}
\begin{Highlighting}[]
\CommentTok{# Print each of the column names}
\KeywordTok{names}\NormalTok{(CO2ex)}
\end{Highlighting}
\end{Shaded}

\begin{verbatim}
## [1] "Plant"     "Type"      "Treatment" "conc"      "uptake"
\end{verbatim}

\begin{Shaded}
\begin{Highlighting}[]
\CommentTok{# Print just one of the columns, the Treatment column}
\NormalTok{CO2ex}\OperatorTok{$}\NormalTok{Treatment}
\end{Highlighting}
\end{Shaded}

\begin{verbatim}
##  [1] nonchilled nonchilled nonchilled nonchilled nonchilled nonchilled
##  [7] nonchilled nonchilled nonchilled nonchilled nonchilled nonchilled
## [13] nonchilled nonchilled nonchilled nonchilled nonchilled nonchilled
## [19] nonchilled nonchilled nonchilled chilled    chilled    chilled   
## [25] chilled    chilled    chilled    chilled    chilled    chilled   
## [31] chilled    chilled    chilled    chilled    chilled    chilled   
## [37] chilled    chilled    chilled    chilled    chilled    chilled   
## [43] nonchilled nonchilled nonchilled nonchilled nonchilled nonchilled
## [49] nonchilled nonchilled nonchilled nonchilled nonchilled nonchilled
## [55] nonchilled nonchilled nonchilled nonchilled nonchilled nonchilled
## [61] nonchilled nonchilled nonchilled chilled    chilled    chilled   
## [67] chilled    chilled    chilled    chilled    chilled    chilled   
## [73] chilled    chilled    chilled    chilled    chilled    chilled   
## [79] chilled    chilled    chilled    chilled    chilled    chilled   
## Levels: nonchilled chilled
\end{verbatim}

\begin{Shaded}
\begin{Highlighting}[]
\CommentTok{# Print the 77th value in the uptake column}
\NormalTok{CO2ex}\OperatorTok{$}\NormalTok{uptake[}\DecValTok{77}\NormalTok{]}
\end{Highlighting}
\end{Shaded}

\begin{verbatim}
## [1] 14.4
\end{verbatim}

\begin{Shaded}
\begin{Highlighting}[]
\CommentTok{# Display the first few rows of the dataframe}
\KeywordTok{head}\NormalTok{(CO2ex)}
\end{Highlighting}
\end{Shaded}

\begin{verbatim}
## Grouped Data: uptake ~ conc | Plant
##   Plant   Type  Treatment conc uptake
## 1   Qn1 Quebec nonchilled   95   16.0
## 2   Qn1 Quebec nonchilled  175   30.4
## 3   Qn1 Quebec nonchilled  250   34.8
## 4   Qn1 Quebec nonchilled  350   37.2
## 5   Qn1 Quebec nonchilled  500   35.3
## 6   Qn1 Quebec nonchilled  675   39.2
\end{verbatim}

\begin{Shaded}
\begin{Highlighting}[]
\CommentTok{# Display the value in row 10, column 5}
\NormalTok{CO2ex[}\DecValTok{10}\NormalTok{,}\DecValTok{5}\NormalTok{]}
\end{Highlighting}
\end{Shaded}

\begin{verbatim}
## [1] 37.1
\end{verbatim}

\begin{Shaded}
\begin{Highlighting}[]
\CommentTok{# Find the mean uptake rate}
\KeywordTok{mean}\NormalTok{(CO2ex}\OperatorTok{$}\NormalTok{uptake)}
\end{Highlighting}
\end{Shaded}

\begin{verbatim}
## [1] 27.2131
\end{verbatim}

\begin{Shaded}
\begin{Highlighting}[]
\CommentTok{# An equivalent approach, since uptake is the 5th column:}
\KeywordTok{mean}\NormalTok{(CO2ex[, }\DecValTok{5}\NormalTok{])}
\end{Highlighting}
\end{Shaded}

\begin{verbatim}
## [1] 27.2131
\end{verbatim}

\begin{Shaded}
\begin{Highlighting}[]
\CommentTok{# See what type of data are stored in each column of the dataframe}
\KeywordTok{str}\NormalTok{(CO2ex)}
\end{Highlighting}
\end{Shaded}

\begin{verbatim}
## Classes 'nfnGroupedData', 'nfGroupedData', 'groupedData' and 'data.frame':   84 obs. of  5 variables:
##  $ Plant    : Ord.factor w/ 12 levels "Qn1"<"Qn2"<"Qn3"<..: 1 1 1 1 1 1 1 2 2 2 ...
##  $ Type     : Factor w/ 2 levels "Quebec","Mississippi": 1 1 1 1 1 1 1 1 1 1 ...
##  $ Treatment: Factor w/ 2 levels "nonchilled","chilled": 1 1 1 1 1 1 1 1 1 1 ...
##  $ conc     : num  95 175 250 350 500 675 1000 95 175 250 ...
##  $ uptake   : num  16 30.4 34.8 37.2 35.3 39.2 39.7 13.6 27.3 37.1 ...
##  - attr(*, "formula")=Class 'formula'  language uptake ~ conc | Plant
##   .. ..- attr(*, ".Environment")=<environment: R_EmptyEnv> 
##  - attr(*, "outer")=Class 'formula'  language ~Treatment * Type
##   .. ..- attr(*, ".Environment")=<environment: R_EmptyEnv> 
##  - attr(*, "labels")=List of 2
##   ..$ x: chr "Ambient carbon dioxide concentration"
##   ..$ y: chr "CO2 uptake rate"
##  - attr(*, "units")=List of 2
##   ..$ x: chr "(uL/L)"
##   ..$ y: chr "(umol/m^2 s)"
\end{verbatim}

\textcolor{red}{Q6: Print out the entry in the 10th row and 3rd column of the subsetted Quebec data}

\begin{Shaded}
\begin{Highlighting}[]
\NormalTok{quebecDat <-}\StringTok{ }\NormalTok{CO2ex }\OperatorTok
\StringTok{  }\KeywordTok{filter}\NormalTok{(Type }\OperatorTok{==}\StringTok{ "Quebec"}\NormalTok{)}
\NormalTok{quebecDat[}\DecValTok{10}\NormalTok{,}\DecValTok{4}\NormalTok{]}
\end{Highlighting}
\end{Shaded}

\begin{verbatim}
## [1] 250
\end{verbatim}

We might also want to filter out particular records of interest. For
instance, if we only want the entries in the dataframe that came from
Quebec, we can pull these records out and save them as a new dataframe,
using the \texttt{subset} function. In the \texttt{subset} function, the
first argument is the name of the original dataframe (CO2ex), and the
second argument is the column you want to use to filter the data (Type),
followed by which entries you want to include. When the filtering is
done on non-numeric data, you'll want to use quotations.

\begin{Shaded}
\begin{Highlighting}[]
\NormalTok{quebecData <-}\StringTok{ }\KeywordTok{subset}\NormalTok{(CO2ex, Type }\OperatorTok{==}\StringTok{"Quebec"}\NormalTok{)}
\CommentTok{# print out to make sure the subset worked}
\NormalTok{quebecData}
\end{Highlighting}
\end{Shaded}

\begin{verbatim}
## Grouped Data: uptake ~ conc | Plant
##    Plant   Type  Treatment conc uptake
## 1    Qn1 Quebec nonchilled   95   16.0
## 2    Qn1 Quebec nonchilled  175   30.4
## 3    Qn1 Quebec nonchilled  250   34.8
## 4    Qn1 Quebec nonchilled  350   37.2
## 5    Qn1 Quebec nonchilled  500   35.3
## 6    Qn1 Quebec nonchilled  675   39.2
## 7    Qn1 Quebec nonchilled 1000   39.7
## 8    Qn2 Quebec nonchilled   95   13.6
## 9    Qn2 Quebec nonchilled  175   27.3
## 10   Qn2 Quebec nonchilled  250   37.1
## 11   Qn2 Quebec nonchilled  350   41.8
## 12   Qn2 Quebec nonchilled  500   40.6
## 13   Qn2 Quebec nonchilled  675   41.4
## 14   Qn2 Quebec nonchilled 1000   44.3
## 15   Qn3 Quebec nonchilled   95   16.2
## 16   Qn3 Quebec nonchilled  175   32.4
## 17   Qn3 Quebec nonchilled  250   40.3
## 18   Qn3 Quebec nonchilled  350   42.1
## 19   Qn3 Quebec nonchilled  500   42.9
## 20   Qn3 Quebec nonchilled  675   43.9
## 21   Qn3 Quebec nonchilled 1000   45.5
## 22   Qc1 Quebec    chilled   95   14.2
## 23   Qc1 Quebec    chilled  175   24.1
## 24   Qc1 Quebec    chilled  250   30.3
## 25   Qc1 Quebec    chilled  350   34.6
## 26   Qc1 Quebec    chilled  500   32.5
## 27   Qc1 Quebec    chilled  675   35.4
## 28   Qc1 Quebec    chilled 1000   38.7
## 29   Qc2 Quebec    chilled   95    9.3
## 30   Qc2 Quebec    chilled  175   27.3
## 31   Qc2 Quebec    chilled  250   35.0
## 32   Qc2 Quebec    chilled  350   38.8
## 33   Qc2 Quebec    chilled  500   38.6
## 34   Qc2 Quebec    chilled  675   37.5
## 35   Qc2 Quebec    chilled 1000   42.4
## 36   Qc3 Quebec    chilled   95   15.1
## 37   Qc3 Quebec    chilled  175   21.0
## 38   Qc3 Quebec    chilled  250   38.1
## 39   Qc3 Quebec    chilled  350   34.0
## 40   Qc3 Quebec    chilled  500   38.9
## 41   Qc3 Quebec    chilled  675   39.6
## 42   Qc3 Quebec    chilled 1000   41.4
\end{verbatim}

Note that in the above code we used a new notation \(==\). If you
remember before we used \textless{}- or equivalently \(=\) to define a
value for an object. For example:

\begin{Shaded}
\begin{Highlighting}[]
\NormalTok{x =}\StringTok{ }\DecValTok{54}
\end{Highlighting}
\end{Shaded}

This defines x to be equal to 54. In the notation above, we instead used
\(==\). Instead of defining a value, this notation (\(==\)) ask a
question: ``Is it equal to?''.

So we could instead ask:

\begin{Shaded}
\begin{Highlighting}[]
\NormalTok{x }\OperatorTok{==}\DecValTok{54}
\end{Highlighting}
\end{Shaded}

\begin{verbatim}
## [1] TRUE
\end{verbatim}

You can see here, that R answers that ``yes'' in fact x is equal to 54.
Alternatively\ldots{}

\begin{Shaded}
\begin{Highlighting}[]
\NormalTok{x }\OperatorTok{==}\StringTok{ }\DecValTok{10}
\end{Highlighting}
\end{Shaded}

\begin{verbatim}
## [1] FALSE
\end{verbatim}

So inside the \(subset()\) command we are asking for only those records
where \(type=="Quebec"\) is true.

Another example, filtering by data that were in Mississippi and have an
uptake rate less than 12

\begin{Shaded}
\begin{Highlighting}[]
\NormalTok{Miss.LowUptake <-}\StringTok{ }\KeywordTok{subset}\NormalTok{(CO2ex, Type }\OperatorTok{==}\StringTok{ "Mississippi"} \OperatorTok{&}\StringTok{ }\NormalTok{uptake }\OperatorTok{<}\StringTok{ }\DecValTok{12}\NormalTok{)}
\NormalTok{Miss.LowUptake}
\end{Highlighting}
\end{Shaded}

\begin{verbatim}
## Grouped Data: uptake ~ conc | Plant
##    Plant        Type  Treatment conc uptake
## 43   Mn1 Mississippi nonchilled   95   10.6
## 57   Mn3 Mississippi nonchilled   95   11.3
## 64   Mc1 Mississippi    chilled   95   10.5
## 71   Mc2 Mississippi    chilled   95    7.7
## 72   Mc2 Mississippi    chilled  175   11.4
## 78   Mc3 Mississippi    chilled   95   10.6
\end{verbatim}

\textcolor{red}{Q7: Filter out data from entries that were in the "chilled" treatment and had an ambient concentration of more than 500. Save this it as a new dataframe. Use `str` or `nrow` to see how many records there are in this new dataframe.}

\begin{Shaded}
\begin{Highlighting}[]
\NormalTok{missChilled <-}\StringTok{ }\NormalTok{Miss.LowUptake }\OperatorTok
\StringTok{  }\KeywordTok{filter}\NormalTok{(Treatment }\OperatorTok{==}\StringTok{ "chilled"} \OperatorTok{&}\StringTok{ }\NormalTok{conc }\OperatorTok{>}\StringTok{ }\DecValTok{500}\NormalTok{)}
\end{Highlighting}
\end{Shaded}

\emph{According to the newly subsetted data fram where the treatment was
``chilled'' and the concentration was above 500, there were no data
entries that matched these criteria}

Syntax is extremely important in R and in the beginning, you will likely
be producing some frustrating error messages. The use of parentheses,
capital letters, commas, etc. makes a big difference in how code is
evaluated. Unfortunately the error messages in R don't always help
diagnose the problem. It's good to get practice interpreting these
errors and de-bugging your code! The lack of an error code doesn't mean
your code ran correctly, either. It is good practice to view data
frames, vectors, plots, etc. after you perform some type of command to
make sure it worked the way you intended it to.

\textcolor{red}{Q8: Debug the following lines of code so that they perform the requested function. To do this, you must first uncomment the lines of code (not the comments themselves) by deleting the pound sign. Then run the line of code, correct the errors, and re-run to make sure it worked.}

\begin{Shaded}
\begin{Highlighting}[]
\CommentTok{# 1. View the first few lines of code}
\KeywordTok{head}\NormalTok{(CO2ex)}
\end{Highlighting}
\end{Shaded}

\begin{verbatim}
## Grouped Data: uptake ~ conc | Plant
##   Plant   Type  Treatment conc uptake
## 1   Qn1 Quebec nonchilled   95   16.0
## 2   Qn1 Quebec nonchilled  175   30.4
## 3   Qn1 Quebec nonchilled  250   34.8
## 4   Qn1 Quebec nonchilled  350   37.2
## 5   Qn1 Quebec nonchilled  500   35.3
## 6   Qn1 Quebec nonchilled  675   39.2
\end{verbatim}

\begin{Shaded}
\begin{Highlighting}[]
\CommentTok{# 2. Subset data from Mississippi}
\CommentTok{# Hint: print out the subsetted data frame. Did it work?}
\NormalTok{missData <-}\StringTok{ }\KeywordTok{subset}\NormalTok{(CO2ex, Type}\OperatorTok{==}\StringTok{"Mississippi"}\NormalTok{)}
\NormalTok{missData}
\end{Highlighting}
\end{Shaded}

\begin{verbatim}
## Grouped Data: uptake ~ conc | Plant
##    Plant        Type  Treatment conc uptake
## 43   Mn1 Mississippi nonchilled   95   10.6
## 44   Mn1 Mississippi nonchilled  175   19.2
## 45   Mn1 Mississippi nonchilled  250   26.2
## 46   Mn1 Mississippi nonchilled  350   30.0
## 47   Mn1 Mississippi nonchilled  500   30.9
## 48   Mn1 Mississippi nonchilled  675   32.4
## 49   Mn1 Mississippi nonchilled 1000   35.5
## 50   Mn2 Mississippi nonchilled   95   12.0
## 51   Mn2 Mississippi nonchilled  175   22.0
## 52   Mn2 Mississippi nonchilled  250   30.6
## 53   Mn2 Mississippi nonchilled  350   31.8
## 54   Mn2 Mississippi nonchilled  500   32.4
## 55   Mn2 Mississippi nonchilled  675   31.1
## 56   Mn2 Mississippi nonchilled 1000   31.5
## 57   Mn3 Mississippi nonchilled   95   11.3
## 58   Mn3 Mississippi nonchilled  175   19.4
## 59   Mn3 Mississippi nonchilled  250   25.8
## 60   Mn3 Mississippi nonchilled  350   27.9
## 61   Mn3 Mississippi nonchilled  500   28.5
## 62   Mn3 Mississippi nonchilled  675   28.1
## 63   Mn3 Mississippi nonchilled 1000   27.8
## 64   Mc1 Mississippi    chilled   95   10.5
## 65   Mc1 Mississippi    chilled  175   14.9
## 66   Mc1 Mississippi    chilled  250   18.1
## 67   Mc1 Mississippi    chilled  350   18.9
## 68   Mc1 Mississippi    chilled  500   19.5
## 69   Mc1 Mississippi    chilled  675   22.2
## 70   Mc1 Mississippi    chilled 1000   21.9
## 71   Mc2 Mississippi    chilled   95    7.7
## 72   Mc2 Mississippi    chilled  175   11.4
## 73   Mc2 Mississippi    chilled  250   12.3
## 74   Mc2 Mississippi    chilled  350   13.0
## 75   Mc2 Mississippi    chilled  500   12.5
## 76   Mc2 Mississippi    chilled  675   13.7
## 77   Mc2 Mississippi    chilled 1000   14.4
## 78   Mc3 Mississippi    chilled   95   10.6
## 79   Mc3 Mississippi    chilled  175   18.0
## 80   Mc3 Mississippi    chilled  250   17.9
## 81   Mc3 Mississippi    chilled  350   17.9
## 82   Mc3 Mississippi    chilled  500   17.9
## 83   Mc3 Mississippi    chilled  675   18.9
## 84   Mc3 Mississippi    chilled 1000   19.9
\end{verbatim}

\begin{Shaded}
\begin{Highlighting}[]
\CommentTok{# 3. Extract the 50th row of the C02ex dataframe and save it as a new vector}
\NormalTok{fifty <-}\StringTok{ }\NormalTok{CO2ex[}\DecValTok{50}\NormalTok{,]}
\NormalTok{fifty}
\end{Highlighting}
\end{Shaded}

\begin{verbatim}
## Grouped Data: uptake ~ conc | Plant
##    Plant        Type  Treatment conc uptake
## 50   Mn2 Mississippi nonchilled   95     12
\end{verbatim}

\hypertarget{loading-data}{%
\section{Loading data}\label{loading-data}}

Often you will want to load your own data into R. R can read .csv and
.txt files but not .xls files. We will read in the file Pelts.csv, which
can be found on D2L. This file contains a list of the number of recorded
pelts collected by the Hudson Bay Company from \(1752\) to \(1819\).

To input the data into R, we will use the function \(read.csv()\). To
read a csv file, it first has to be saved into your working directory,
which is likely the same directory (or folder) as this R Markdown file
(unless you have changed it). To see what R is using as your working
directory, run the command \texttt{getwd}:

\begin{Shaded}
\begin{Highlighting}[]
\KeywordTok{getwd}\NormalTok{()}
\end{Highlighting}
\end{Shaded}

\begin{verbatim}
## [1] "/Users/jacksonanderson/School/BioStats/MarkdownFiles"
\end{verbatim}

Navigate to this folder on your computer and make sure Pelts.csv is in
there. After you have made sure the dataset is in the same folder, you
can load the csv file using \(read.csv()\).You want to assign the loaded
data to an object in R so you can work with it. For example, you could
type \(pelts <- read.csv("Pelts.csv")\).

Do not use \texttt{file.choose()} with Rmarkdown documents as it will
impair knitting. Get in the habit of organizing your files and using
\texttt{read.csv()}!

A good shortcut to see the files currently in your working directory, is
to go to the files tab in the panel to the right. You can hit the
\texttt{More} menu and select ``go to working directory'' and view the
files currently in that folder.

Read in Pelts.csv and save it as an object.

\begin{Shaded}
\begin{Highlighting}[]
\NormalTok{peltsDat <-}\StringTok{ }\KeywordTok{read_csv}\NormalTok{(}\StringTok{"/Users/jacksonanderson/School/BioStats/Data/Pelts.csv"}\NormalTok{)}
\end{Highlighting}
\end{Shaded}

\begin{verbatim}
## Parsed with column specification:
## cols(
##   date = col_double(),
##   no.pelts = col_double()
## )
\end{verbatim}

\begin{Shaded}
\begin{Highlighting}[]
\CommentTok{#was having trouble setting my working directory and then calling the csv using a code chunk, so I just used the path name}
\end{Highlighting}
\end{Shaded}

Examine what type of data Pelts contains by using the \emph{class()} or
\emph{str()} functions. The function \emph{summary()} also gives a nice
overview of your data. Try all three out.

\begin{Shaded}
\begin{Highlighting}[]
\KeywordTok{summary}\NormalTok{(peltsDat)}
\end{Highlighting}
\end{Shaded}

\begin{verbatim}
##       date         no.pelts     
##  Min.   :1752   Min.   : 116.0  
##  1st Qu.:1769   1st Qu.: 986.8  
##  Median :1786   Median :1551.5  
##  Mean   :1786   Mean   :2137.3  
##  3rd Qu.:1802   3rd Qu.:2858.8  
##  Max.   :1819   Max.   :7179.0
\end{verbatim}

\begin{Shaded}
\begin{Highlighting}[]
\KeywordTok{str}\NormalTok{(peltsDat)}
\end{Highlighting}
\end{Shaded}

\begin{verbatim}
## Classes 'spec_tbl_df', 'tbl_df', 'tbl' and 'data.frame': 68 obs. of  2 variables:
##  $ date    : num  1752 1753 1754 1755 1756 ...
##  $ no.pelts: num  4009 7179 4198 1444 838 ...
##  - attr(*, "spec")=
##   .. cols(
##   ..   date = col_double(),
##   ..   no.pelts = col_double()
##   .. )
\end{verbatim}

\begin{Shaded}
\begin{Highlighting}[]
\KeywordTok{class}\NormalTok{(peltsDat)}
\end{Highlighting}
\end{Shaded}

\begin{verbatim}
## [1] "spec_tbl_df" "tbl_df"      "tbl"         "data.frame"
\end{verbatim}

\textcolor{red}{Q9: What type of object is pelts? What other information do you get from using the $str()$ function?}

\emph{Your answer here}

\textcolor{red}{Q10: Calculate the mean number of pelts harvested between $1752$ and $1761$. Use any of the functions we've gone over already (like the subset, length, mean, or sum functions). Store your answer as an object and print it. Comment your code so that a reader could follow your logic.}

\begin{Shaded}
\begin{Highlighting}[]
\NormalTok{mean <-}\StringTok{ }\NormalTok{peltsDat }\OperatorTok
\StringTok{  }\KeywordTok{filter}\NormalTok{(date }\OperatorTok{>}\StringTok{ }\DecValTok{1752} \OperatorTok{&}\StringTok{ }\NormalTok{date }\OperatorTok{<}\StringTok{ }\DecValTok{1761}\NormalTok{) }\OperatorTok\StringTok{ }\CommentTok{#subsetting data by data range}
\StringTok{  }\KeywordTok{summarise}\NormalTok{(}\KeywordTok{mean}\NormalTok{(no.pelts)) }\CommentTok{#using the summarise function in dplyr to calculate mean of entire no.pelts vector}
\end{Highlighting}
\end{Shaded}

I have also placed a copy of this file on GitHub. You can also load data
from a csv file at a specific URL using \(read.csv()\)

\begin{Shaded}
\begin{Highlighting}[]
\NormalTok{pelts_web<-}\KeywordTok{read.csv}\NormalTok{(}\StringTok{"https://raw.githubusercontent.com/StatsTree/Datasets/master/Pelts.csv"}\NormalTok{)}
\end{Highlighting}
\end{Shaded}

\hypertarget{making-your-own-data-frame}{%
\section{Making your own data frame}\label{making-your-own-data-frame}}

Often you will load in previously compiled data (say, from an excel
workbook converted to .csv). Sometimes you will want to create your own
data frames from scratch in R. We will start by creating a few different
vectors and then combine them into a dataframe. Here is an example of a
dataframe with 3 columns:

\begin{Shaded}
\begin{Highlighting}[]
\NormalTok{fish_type<-}\StringTok{ }\KeywordTok{c}\NormalTok{(}\StringTok{"lake trout"}\NormalTok{, }\StringTok{"brown trout"}\NormalTok{, }\StringTok{"lake trout"}\NormalTok{, }\StringTok{"rainbow trout"}\NormalTok{, }\StringTok{"brown trout"}\NormalTok{)}
\NormalTok{parasite<-}\StringTok{ }\KeywordTok{c}\NormalTok{(}\StringTok{"yes"}\NormalTok{, }\StringTok{"no"}\NormalTok{, }\StringTok{"yes"}\NormalTok{, }\StringTok{"yes"}\NormalTok{, }\StringTok{"no"}\NormalTok{)}
\NormalTok{length_fish<-}\StringTok{ }\KeywordTok{c}\NormalTok{(}\DecValTok{40}\NormalTok{, }\DecValTok{63}\NormalTok{, }\DecValTok{48}\NormalTok{, }\DecValTok{51}\NormalTok{, }\DecValTok{69}\NormalTok{)}
\NormalTok{df<-}\KeywordTok{data.frame}\NormalTok{(fish_type,parasite,length_fish)}
\KeywordTok{str}\NormalTok{(df)}
\end{Highlighting}
\end{Shaded}

\begin{verbatim}
## 'data.frame':    5 obs. of  3 variables:
##  $ fish_type  : Factor w/ 3 levels "brown trout",..: 2 1 2 3 1
##  $ parasite   : Factor w/ 2 levels "no","yes": 2 1 2 2 1
##  $ length_fish: num  40 63 48 51 69
\end{verbatim}

\begin{Shaded}
\begin{Highlighting}[]
\KeywordTok{head}\NormalTok{(df)}
\end{Highlighting}
\end{Shaded}

\begin{verbatim}
##       fish_type parasite length_fish
## 1    lake trout      yes          40
## 2   brown trout       no          63
## 3    lake trout      yes          48
## 4 rainbow trout      yes          51
## 5   brown trout       no          69
\end{verbatim}

\hypertarget{knitting-your-rmd-document}{%
\section{Knitting your Rmd document}\label{knitting-your-rmd-document}}

You will turn in your biometry assignments in as both an .Rmd document
(like the one you are currently working in) and a PDF. The PDF version
is a nicely formatted document which will display results and graphics.
The process of making the PDF version is called ``knitting''. Try
kitting the current document by hitting the icon that says ``knit'' in
the menu bar above, or choosing ``knit document'' from the File menu.

If there are errors in your code, the knitting process does not work. If
you get an error message, navigate to the line listed and de-bug the
code.

Still not working? If you are at the end of your rope and cannot fix
broken code (and you've already tried meeting with your TA!), you may
``comment out'' the line that does not work using a \# before the line
of code, and then knit the PDF.

\end{document}
